% ****** Start of file apssamp.tex ******
%
%   This file is part of the APS files in the REVTeX 4.2 distribution.
%   Version 4.2a of REVTeX, December 2014
%
%   Copyright (c) 2014 The American Physical Society.
%
%   See the REVTeX 4 README file for restrictions and more information.
%
% TeX'ing this file requires that you have AMS-LaTeX 2.0 installed
% as well as the rest of the prerequisites for REVTeX 4.2
%
% See the REVTeX 4 README file
% It also requires running BibTeX. The commands are as follows:
%
%  1)  latex apssamp.tex
%  2)  bibtex apssamp
%  3)  latex apssamp.tex
%  4)  latex apssamp.tex
%
\documentclass[%
%%%%%%%% reprint,
%superscriptaddress,
%groupedaddress,
%unsortedaddress,
%runinaddress,
%frontmatterverbose, 
%preprint,
%preprintnumbers,
%nofootinbib,
%nobibnotes,
%bibnotes,
 amsmath,amssymb,
 aps,
%pra,
prb,
%rmp,
%prstab,
%prstper,
floatfix, showkeys
%%%%%%%%%%%%%%%%%%%%%  , twocolumn
]{revtex4-2}
\usepackage{graphicx}% Include figure files
\usepackage{dcolumn}% Align table columns on decimal point
\usepackage{bm}% bold math
\usepackage{epsfig}
\usepackage{appendix}
%\usepackage{xcolor}
\usepackage{dcolumn}% Align table columns on decimal point
\usepackage{bm}% bold math
%\usepackage{color}
%%%%%%%%%\usepackage{cite}
\usepackage{float}
%\usepackage{float}
%\usepackage{float}
\usepackage{parskip} %% <-- added 
\usepackage[section]{placeins}
\usepackage{silence}
\WarningFilter{revtex4-2}{Repair the float package}
%\usepackage{multicol}
\newcommand{\need}[1]{\textcolor{red}{#1}}
\newcommand{\modif}[1]{\textcolor{blue}{#1}}
%\newcommand{\modif}[1]{\textcolor{black}{#1}}
%\newcommand{\need}[1]{\textcolor{black}{#1}}
%\newcommand{\mod}[1]{\textcolor{black}{#1}}
\newcommand{\mage}[1]{\textcolor{magenta}{#1}}
\newcommand{\green}[1]{\textcolor{green}{#1}}
\newcommand{\olive}[1]{\textcolor{olive}{#1}}
\newcommand{\sign}{\mathop{\mathrm{sign}}}
%\newcommand{\eff}{{\textsc{\tiny eff}}}
\newcommand{\eff}{{\mbox{\small eff}}}
\newcommand{\comma}{, }
\newcommand{\add}[1]{\textcolor{addcolor}{#1}}
\newcommand{\delete}[1]{\textcolor{deletecolor}{\sout{#1}}}
\newcommand{\replace}[2]{\textcolor{deletecolor}{\sout{#1}} \textcolor{addcolor}{#2}}

\usepackage[dvipsnames]{xcolor}
\newcommand{\Xopt}{X_{\mathrm{opt}}}
\newcommand{\RRS}{R_{\mathrm{RS}}}
\newcommand{\dd}{\mathrm{d}}                     % Differential
\newcommand{\ee}{\mathrm{e}}                     % Euler number
\newcommand{\ii}{\mathrm{i}}                     % Imaginary unit
\newcommand{\pd}[2]{\frac{\partial #1}{\partial #2}}
\newcommand{\ddt}[2][]{\frac{\dd #1}{\dd #2}}
\newcommand{\vect}[1]{\boldsymbol{#1}}           % 3-vector
\newcommand{\ten}[1]{\mathbf{#1}}                % Rank-2 tensor
\newcommand{\avg}[1]{\langle #1\rangle}          % Expectation value
\newcommand{\abs}[1]{\left| #1 \right|}             % Absolute value
\newcommand{\order}[1]{\mathcal{O}\!\left(#1\right)}
%\newcommand{\mycustomline}
\renewcommand{\thesubsection}{\thesection.\arabic{subsection}}
\renewcommand{\thesubsubsection}{\thesubsection\Alph{subsubsection}}
\renewcommand{\andname}{\ignorespaces}
%%%%%%%%%%%%%%%%%%%%%%%\usepackage{ragged2e}


%\usepackage{hyperref}% add hypertext capabilities
%\usepackage[mathlines]{lineno}% Enable numbering of text and display math
%\linenumbers\relax % Commence numbering lines

%\usepackage[showframe,%Uncomment any one of the following lines to test 
%%scale=0.7, marginratio={1:1, 2:3}, ignoreall,% default settings
%%text={7in,10in},centering,
%%margin=1.5in,
%%total={6.5in,8.75in}, top=1.2in, left=0.9in, includefoot,
%%height=10in,a5paper,hmargin={3cm,0.8in},
%]{geometry}

\begin{document}

% \preprint{APS/123-QED}

\title{{\modif{ Parameter--Free Particle Masses from a $\varphi$--Sheet Fixed Point }}}
%Manuscript Title:\\with Forced Linebreak}% Force line breaks with \\
% \thanks{A footnote to the article title}%

\author{Jonathan Washburn}
% \altaffiliation[Also at ]{Physics Department, XYZ University.}%Lines break automatically or can be forced with \\
%\author{Elshad Allahyarov}%
% \email{elshad.allakhyarov@case.edu}
\affiliation{%
{ \it  Recognition Physics Institute, Austin TX, USA}
%  \\  This line break forced with \textbackslash\textbackslash
}%

%\collaboration{MUSO Collaboration}%\noaffiliation

\author{Elshad Allahyarov}
% \homepage{http://www.Second.institution.edu/~Charlie.Author}
%\RaggedRight
\affiliation{ \begin{flushleft} 1. Recognition Physics Institute\comma   Austin TX\comma  USA  \end{flushleft}}
\affiliation{  \begin{flushleft} 2. Institut f\"ur Theoretische Physik II: Weiche Materie\comma
  HHU D\"usseldorf\comma  Universit\"atstrasse 1\comma 40225 D\"usseldorf\comma Germany \end{flushleft}}
\affiliation{ \begin{flushleft} 3. Theoretical Department\comma JIHT RAS (OIVTAN)\comma
    13/19 Izhorskaya street\comma Moscow 125412\comma Russia \end{flushleft}}%
\affiliation{ \begin{flushleft} 4. Department of Physics\comma Case Western Reserve University\comma Cleveland\comma Ohio 44106-7202
    \comma USA \end{flushleft}}

%\collaboration{CLEO Collaboration}%\noaffiliation

\date{\today}% It is always \today, today,
             %  but any date may be explicitly specified












\begin{abstract}
We develop a parameter–free framework that predicts Standard Model masses and mixings by solving a rung–indexed $\varphi$–ladder as a local fixed point and then replacing the arbitrary probe scale with a signed, $\ell_1$–normalized $\varphi$–sheet average tied to the same alternating gap series that defines the ledger. The only inputs are physical constants and inclusive $e^+e^-\!\to\!\text{hadrons}$ information used in a dispersion evaluation of $\alpha_{\rm em}(\mu)$. Charged–lepton ratios are reproduced at parts–per–million (ppm) once the hadronic vacuum–polarization integral is numerically densified in the $\tau$ window; the solver, invariants, and running layer remain unchanged. A single global scale set from the atmospheric neutrino splitting fixes absolute Dirac neutrino masses with $\Sigma m_\nu\simeq 0.0605$ eV and transfers to $(e,\mu,\tau)$ in eV at ppm precision. We also provide a fully internal absolute unit from a $Z/W$ identity with a ledger–driven tilt, eliminating external masses and $\Delta m^2$ inputs. The boson sector reproduces $Z/W$ and $H/Z$ at the $10^{-3}$ level, and quark mass ratios—formed "$\varphi"–fixed'' at each species' self–consistent scale $\mu_\star$—agree at the few$\times10^{-3}$ level. CKM and PMNS magnitudes follow from the same rung geometry without new parameters. We supply a reproducible pipeline and a compact, quantitative error budget that attributes the residuals to dispersion quadrature density in the $\tau$ window and fixed–point stability.\par\medskip
\end{abstract}

\keywords{ {\modif{
axiomatic physics, type theory, foundations of physics, logical necessity,  tautology, dark matter, cosmology}
}}


\maketitle
%\tableofcontents
% Notation \& Definitions (quick reference)





%\twocolumngrid
\onecolumngrid

\modif{
  \section{ Introduction}
\label{sec-1}
}
The Standard Model (SM) \cite{SM-ref,weinberg-book,Weinberg1979} achieves extraordinary accuracy across many decades in energy, yet its numerical content is carried by dozens of a priori free inputs—chiefly particle masses and mixing parameters \cite{PDG2022,PDG2025}. Absent tuned textures or auxiliary flavour symmetries, there is no accepted predictive mechanism for the values themselves. Numerous extensions—supersymmetry \cite{dine-1993,Wess1974}, technicolour \cite{Susskind1979,hill-2003,technicolor-2015}, extra dimensions \cite{Randall1999}, GUTs \cite{grand-uni-th-2015}, loop quantum gravity \cite{Rovelli2004,loop-qg}, and string theory \cite{polchinski-1998}—have not resolved the hierarchy and typically introduce additional knobs. Phenomenological constructions (e.g. Froggatt–Nielsen \cite{frog-1979,fritz-2000} or modular approaches \cite{petcov}) organize structure but do not fix absolute masses. A few numerological proposals \cite{koide-1983,eln-2002,eln-2002-1,cascade-2003} arrange patterns but rely on ad hoc rescalings.
\par
This work presents a minimal, measurement–anchored alternative. Each species $i$ obeys the mass law
\begin{equation}
  m_i \;=\; B_i\,E_{\rm coh}\;\varphi^{\,r_i+f_i(\ln m_i)}\,,\qquad r_i\in\mathbb{Z}\,,
  \label{eq:mass_law_intro}
\end{equation}
where $r_i$ is an integer\,rung on a $\varphi$–ladder, $B_i$ is a fixed sector factor, $E_{\rm coh}$ a common normalization, and the small residue $f_i$ decomposes into (i) a scale–window average of the usual anomalous dimension $\gamma_i(\mu)$ and (ii) a rung–dependent gap series built from fixed ledger invariants. Masses are defined nonperturbatively as 
solutions of a local $\varphi$–cycle fixed point
\begin{equation}
  \ln m_i = \ln(B_iE_{\rm coh}) + r_i\ln\varphi + f_i(\ln m_i)\,\ln\varphi\,.
\end{equation}
To remove the probe–scale ambiguity, we replace the single window by a signed, $\ell_1$–normalized $\varphi$–sheet average over adjacent windows with weights inherited from the ledger's alternating gap coefficients. The weights have a closed–form normalizer $2\ln\varphi$ and the tail admits a supergeometric bound, rendering the truncation a purely numerical tolerance.
\par
Running inputs are standard and shared across sectors. For charged leptons, $\gamma_i(\mu)=\gamma^{\rm QED}_i(\mu)+\gamma^{\rm SM}_i(\mu)$ with the QED piece evaluated at $\alpha_{\rm em}(\mu)$ obtained from a Euclidean dispersion relation for hadronic vacuum polarization \cite{EidelmanJegerlehner1995,Jegerlehner2003,Keshavarzi2019,Davier2017}. Electroweak running adopts two–loop gauge mixing with GUT normalization for $g_1$ \cite{MachacekVaughn1983-85,Buttazzo2013}. Quark ratios are compared "$\varphi"–fixed'' by evolving each mass to its self–consistent scale $\mu_\star$ \cite{ChetyrkinKuehnSteinhauser2000,HerrenSteinhauser2018}. No sector parameters or fitted weights are introduced; ledger invariants are fixed and injected solely via the rung.
\par
Empirically, the same rung–locked ledger spans leptons, neutrinos, W/Z/H, and quarks while reproducing CKM/PMNS magnitudes without additional texture assumptions. Charged–lepton ratios are matched at the ppm level once the $\tau$ window in the dispersion integral is numerically densified (the architecture and solver remain unchanged). A single global scale set by $\Delta m^2_{31}$ yields absolute Dirac neutrino masses with $\Sigma m_\nu\simeq 0.0605$ eV, and an internal $Z/W$ identity provides a fully parameter–free absolute unit that removes external mass and $\Delta m^2$ inputs. Boson ratios $Z/W$ and $H/Z$ are reproduced at $\sim10^{-3}$; quark ratios agree at the few$\times10^{-3}$ level when compared at $\mu_\star$. We quantify residuals and stability in a compact error budget.
\par
The paper proceeds as follows. Section~\ref{sec:formalism} formalizes the mass law, the local $\varphi$–cycle fixed point, and the $\varphi$–sheet average; Section~\ref{sec:running} details the running and dispersion inputs; Section~\ref{sec:results} reports cross–sector results and anchors; Section~\ref{sec:error_stability} presents the error budget and stability studies.

\section{Results}\label{sec:results}
\subsection{Charged leptons}
For reference the dimensionless ladder ratios are
\[
\mu/e=206.772097,\quad \tau/\mu=16.818047,\quad \tau/e=3477.584758.
\]
\begin{table}[H]
\caption{Charged--lepton summary (rung $r$, sector factor $B$, fractional residue $f$, experimental and calculated pole masses, and residuals). Residuals are in parts per million (ppm) relative to PDG pole masses.}
\label{tab:leptons}
\begin{tabular}{l c c c c c c}
\hline
Particle & $r$ & $B$ & $f$ & $m_{\rm exp}$ (MeV) & $m_{\rm calc}$ (MeV) & $\delta$ (ppm) \\
\hline
$e$   & 0  & 1 & $1.20\times10^{-3}$ & 0.51099895 & 0.51099895 & $<\!1$ \\
$\mu$ & 11 & 1 & $8.0\times10^{-4}$ & 105.658374 & 105.658374 & $<\!1$ \\
$\tau$& 17 & 1 & $6.0\times10^{-4}$ & 1776.86 & 1776.86 & $<\!100$ \\
\hline
\end{tabular}
\end{table}

\subsection{Dirac neutrinos (NO)}
Absolute masses follow after fixing the single global scale $s$ from the atmospheric splitting; $\Sigma m_\nu\!=\!0.06053$ eV.
\begin{table}[H]
\caption{Dirac neutrino summary (normal ordering). Masses are absolute predictions after fixing $s$ from $\Delta m^2_{31}$.}
\label{tab:neutrinos}
\begin{tabular}{l c c c c l}
\hline
Particle & $r$ & $B$ & $f$ & $m_{\rm calc}$ (meV) & Notes \\
\hline
$\nu_1$ & 7  & 1 & $1.1\times10^{-3}$ & 2.083 & normal ordering \\
$\nu_2$ & 9  & 1 & $0.9\times10^{-3}$ & 9.023 &  \\
$\nu_3$ & 12 & 1 & $0.8\times10^{-3}$ & 49.427 &  \\
\hline
\end{tabular}
\end{table}

\subsection{Bosons}
Locked ratios: $Z/W=1.1332824$, $H/Z=1.3721798$, $H/W=1.5549887$.
\begin{table}[H]
\caption{Boson block anchored to $M_W$. Ratios are rung--gap locked; absolutes follow by anchoring to $M_W$.}
\label{tab:bosons}
\begin{tabular}{l c c c c c c}
\hline
Particle & $r$ & $B$ & Ratio & $m_{\rm calc}$ (GeV) & $m_{\rm exp}$ (GeV) & $\delta$ (\%) \\
\hline
$W$ & 44  & 4 & ---                 & 80.379 (anchor) & 80.379   & 0 \\
$Z$ & --- & 4 & $Z/W=1.1332824$    & 91.0921         & 91.1876  & $-0.105$ \\
$H$ & --- & 4 & $H/Z=1.3721798$    & 124.9947        & 125.10   & $-0.084$ \\
\hline
\end{tabular}
\end{table}

\subsection{Quark sector (\,$\varphi$--fixed)}
Each experimental mass is evolved to its own fixed--point scale $\mu_\star$ before forming ratios.
\begin{table}[H]
\caption{Quark ratios evaluated in the $\varphi$--fixed prescription, i.e. at each species' self--consistent scale $\mu_\star$.}
\label{tab:quarks}
\begin{tabular}{l c c c c c c}
\hline
Sector & Ratio & Predicted & Experimental & $\delta$ (\%) & Notes & $B$ \\
\hline
Down & $s/d$ & 20.1695669   & 20.1052632   & $+0.3198$ & at $\mu^\star$ & 2 \\
Down & $b/s$ & 43.7291176   & 43.7644231   & $-0.0807$ &                & 2 \\
Down & $b/d$ & 881.9961625  & 879.8093108  & $+0.2486$ &                & 2 \\
Up   & $c/u$ & 586.7231268  & 587.9629630  & $-0.2109$ &                & 2 \\
Up   & $t/c$ & 135.8306806  & 135.8267717  & $+0.0029$ &                & 2 \\
Up   & $t/u$ & 79695.5311281 & 79858.8310185 & $-0.2045$ &               & 2 \\
\hline
\end{tabular}
\end{table}


































        let us do the following. rewrite sections 2 and 3, include all elaborations, explain all terminology, all expressions, formula, all parameters, all slengs,  all the anomalous dimensions, high-loop QCD mass AD, g_1 in GUT normalization, up to for 4L, SU(2), gamma_i(\mu), QED mass anomalous dimension, vacuum polarization,  high-Q^2 behaviour, densification of the dispersion quadrature, the \tau window. give me a latef file for sections 2 and 3,  with everything clearly defined and written. please, try your best and avoid any simplifications and shortcuts, i hate that.




        this is from jonathan

Partial empirical (MW/GF anchors)
What we did: Fixed s from a single measured quantity (either the W mass MW or the Fermi constant GF).
Effect: Removed ν-inputs but still used one experimental constant.
Status: Partly empirical.
Z/W identity with SM tilt ("tilt" = cosθW(μ) from SM running)
What we did: Used only dimensionless ladder outputs mWφ, mZφ and solved
(mZφ/mWφ)·cosθW(μ) = 1, then s = μ/mWφ,
with cosθW(μ) from two-loop SM RGE (uses known low-energy couplings).
Effect: No experimental masses or Δm^2; still leans on empirical EW couplings.
Status: Almost internal; "no masses in," but SM running inputs remain.
Final (current default): Z/W identity with RS force-ladder tilt
What we do now: Same identity, but cosθW(μ) is built purely from φ, the ledger gap series g_m, and the RS recognition energy Erec (no external inputs).
Solve F(μ) = (mZφ/mWφ)·cosθW(μ) − 1 = 0 → μ, then s = μ/mWφ.
Effect: No experimental masses, no Δm^2, no low‑energy couplings. Neutrino absolutes become predictions. We also report s/Erec as the bridge constant.
Status: Fully internal, parameter‑free.
One‑line summary of the final process:
Solve (mZφ/mWφ)·cosθW(μ) = 1 with cosθW from the RS force‑ladder map; set s = μ/mWφ.
Reference:
Full paper with the internal Z/W anchor description is here: Parameter‑Free Particle Masses (PDF).
Summary:
ν-anchored → MW/GF‑anchored → Z/W+SM tilt → Z/W+RS force‑ladder tilt (final, fully internal).        
        










        
